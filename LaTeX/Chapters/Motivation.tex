\chapter{Einleitung}\label{ch:Einleitung}
Diese Thesis wird im Rahmen eines Informatik-Studiengangs mit dem Abschluss eines Master of Science geschrieben. Worum es genau geht und welche Bedingungen ich voraussetze, wird im folgenden Kapitel beschrieben.

\section{Motivation}\label{sec:Motivation}
Schaut man in die Natur sieht man sehr häufig, dass sich Tiere in riesigen Gruppen organisieren, die sich meist ohne zentralen Koordinator bewegen.
Bei Fischschwärmen handelt jedes Individuum vollkommen autonom und nur anhand dessen was es in seiner (unmittelbaren) Umgebung erkennen kann. Dennoch können komplexe Handlungen wie die Flucht vor einem Räuber oder die gemeinsame Wanderung zu einem neuen Futterort koordiniert werden.

In manchen Herden dagegen besteht ein 'Schwarm' aus Untergruppen, welche zwar einen Anführer haben, der die Richtung für seine Gruppe angibt, diese Führer sich jedoch am Rest der Herde orientieren und somit ein Schwarmverhalten entsteht, bei dem die Akteure aus koordinierten Einzelgruppen und nicht aus einzelnen Tieren bestehen.
Ebenso wird es bei Vögeln vorgefunden, welche durch ihre Koordination in einer energiesparenden Formation fliegen können.
Solch ein Schwarmverhalten hat den großen Vorteil, dass jeder Akteur nur für seine eigene Bewegung verantwortlich ist. Der Denkaufwand verteilt sich auf alle Mitglieder und die, meist langsame, Kommunikation untereinander wird auf ein Minimum reduziert.

Man fing früh an dieses Verhalten zu studieren und es auf Roboter anwenden zu wollen. Schwärme von Robotern können in vielerlei Hinsicht eingesetzt werden, darunter Sensornetzwerke, die dynamisch einen bestimmten Raum überwachen oder Arbeitsroboter, die ohne weiteres Zutun bestimmte Tätigkeiten wie Transport übernehmen.

Zentrale Systeme mit Koordinator können dafür allerdings schnell unzulänglich werden. Ein großer Nachteil ist schon dadurch gegeben, dass es diesen einen Koordinator geben muss. Fällt er wegen einem technischen Defekt aus, steht der Schwarm still. Baut man einen Backup-Koordinator hat man einen vielfach erhöhten Aufwand beim Design der Schwarm-Architektur, einen größeren Aufwand in der Kommunikation, da der Backup mit einbezogen werden muss, und letztlich auch einen höheren Preis, weil der rechenstarke Computer gleich mehrmals gebaut werden muss.

Auch die Reichweite des zentralen Koordinators kann dem Schwarm schnell zum Verhängnis werden. Hat ein Schwarm Beispielsweise die Aufgabe ein Gebiet zu erkunden um später eine Karte aus den Sensorwerten zu erstellen, muss der Koordinator entweder stets in Reichweite aller Roboter sein um mit ihnen kommunizieren zu können oder er muss die Kommunikation durch teures Flooding im Kommunikationsnetz aufrecht erhalten. Bei der Koordination von Drohnen kann allein die Erreichbarkeit zu einem sehr realen Problem werden, da der Koordinator aufgrund seiner Rechenleistung ein vielfaches der Arbeiter wiegen wird. Dies führt zu beträchtlich steigenden Kosten für die Motoren und den Akku des Koordinators, beeinflusst aber auch sein Verhalten, da er deutlich träger ist als die anderen.
Wer das Bild eines Fischschwarms vor Augen hat, wird vermutlich auch direkt an die Anzahl der Mitglieder des Schwarms denken müssen. Zentrale Systeme sind nur schwer skalierbar und müssen von Anfang an für die jeweilige Größe konzipiert werden.

Schwarmverhalten zeichnet sich dadurch aus, dass die einzelnen Mitglieder des Schwarms ihre Bewegung selbst koordinieren. Dabei wird sich entweder nur nach der Bewegung der anderen gerichtet oder es werden sehr lokal Informationen ausgetauscht. Die Rechenleistung wird auf alle Mitglieder gleichmäßig aufgeteilt und jeder ist nur für sich selbst verantwortlich. Durch die lokale Kommunikation stellt die Größe oder Ausbreitung des Schwarms kein Hindernis dar, da ein Roboter nur Verbindung zu seinen nächsten Nachbarn braucht und auch auf teures Flooding verzichtet werden kann. Die Größenordnung des Schwarms spielt dabei auch eine untergeordnete Rolle, da ein Roboter aufgrund des lokalen Informationsaustauschs ohnehin nur wenige Kommunikationspartner hat. Ebenfalls ist solch ein Schwarm ausfallsicherer da eine ausgefallene Einheit (egal welche) letztlich nur sich selbst stoppt. 

\section{Aufgabenstellung}\label{sec:Aufgabenstellung}
In dieser Thesis befasse ich mich mit der Thematik, einen Roboterschwarm dafür nutzen zu können um Transportaufträge zu bearbeiten. Ein Auftrag wird an beliebig viele Roboter eines Schwarms verteilt, woraufhin diese selbstständig eine Untergruppe bilden, die für die Ausführung des Auftrages zuständig ist. Da die zu transportierenden Objekte beliebiger Größe und Form sein können, gilt es unter Umständen eine Formation zu bilden und diese bis zum Ende des Auftrags einzuhalten. Dabei gilt es Kollisionen mit anderen Objekten zu vermeiden und selbstständig navigieren zu können. Eine eventuell benötigte Karte ist zunächst voreingestellt, wird im späteren Verlauf aber, von bereits im System der Roboter vorhandenen Funktionen, selbstständig erstellt werden können.

Hauptgegenstand dieser Thesis ist die Erarbeitung des Konzepts und die Umsetzung der Koordination bzw. des Verhaltens der Roboter als Algorithmus, sodass dieser produktiv eingesetzt werden kann. Eine Implementierung findet prototypisch auf den Robotern selbst oder einer Simulation dieser statt und dient vor allem dazu festzustellen, ob die theoretischen Überlegungen auch in der Praxis angewendet werden können.

\section{Aufbau der Thesis}\label{sec:AufbauDerThesis}
Zunächst werde ich im Kapitel Grundlagen und Hintergründe auf notwendige Grundlagen zum Thema Schwarmverhalten eingehen. Darunter verschiedene Definitionen, Abgrenzungen und Einstiege in für Schwarmverhalten wichtige Themen.

Danach folgt eine Analyse der Aufgabenstellung im Detail. Es wird darauf eingegangen, welche Probleme sich bei der Aufgabenstellung auftun und die es zu lösen gilt und welche Fähigkeiten bei den Robotern genau gefordert werden.

Im darauf folgenden Kapitel wird bereits vorhandene Forschung zum Thema Schwarmverhalten untersucht und vorgestellt. Diese Forschung wird sich nicht ausschließlich um den Bereich der Roboter-Technik drehen, zumal es für das behandelte Thema noch keine ausreichende Literatur gibt. Die Forschung wird darauf geprüft, welche Themenbereiche meiner Thesis bereits abgedeckt werden und welche noch offen sind.

Als nächstes folgen die Kapitel der Konzeption, Implementierung und Evaluation. In diesen wird das Konzept für die Roboter zunächst theoretisch, dann praktisch entwickelt werden wird um es anschließend Simulationen zu unterziehen, die das Konzept testen und ein Gefühl dafür geben, wie sich der Schwarm unter bestimmten Voraussetzungen verhält. Erkenntnisse aus der prototypischen Entwicklung und der Evaluation werden wieder zurück in das Konzept fließen und dieses verbessern.

Am Ende der Thesis findet eine Evaluation der erreichten Ziele statt und, sollten einige nicht erreicht worden sein, eine Erläuterung woran es gelegen hat. Außerdem wird es einen Überblick darüber geben, was mit mehr Zeit und Ressourcen noch hätte erreicht werden können, bzw. wo man Ansetzen könnte um die Entwicklung weiterzuführen.