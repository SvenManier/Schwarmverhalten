\chapter{Konzeption}\label{ch:Konzeption}

In diesem Kapitel werde ich die Konzeption der Thesis beschreiben und einige kleinere Algorithmen näher erläutern. Generell beruht die Konzeption dieser Thesis auf dem ROS~(Robot Operating System)-Framework, welches jedem quelloffen unter der BSD-Lizenz zur Verfügung steht~\footnote{\url{http://www.ros.org/is-ros-for-me/}}.

\section{Generelles}
Die Konzeption besteht aus iterativen Schritten in denen ein bestimmtes Verhalten erarbeitet und anschließend ausgewertet wird. Dabei wird nicht mit dem erstbesten Ergebnis weiter gemacht, sondern es werden verschiedene Herangehensweisen getestet und für die nächsten Schritte evaluiert. Auch viele Iterationen später könnte sich ein Vorgehen, welches zunächst als unzureichend eingestuft wurde, als das bessere herausstellen.\\

Die Konzeption findet auf Ubuntu LTS 16.04 mit der ROS-Version 'Lunar' statt, da diese beiden Komponenten das derzeit stabilste Duo bilden. Aufgrund mangelnder Kapazitäten, befindet sich das Ubuntu-System in einer Virtuellen Maschine, was, bis auf einen höheren Ressourcen-Verbrauch, allerdings keinerlei erkennbare Nachteile mit sich zieht.

Da ein experimentelles Vorgehen direkt an realen Robotern zu Aufwändig wäre, insbesondere was die Zeitkosten für die Durchführung einer Simulation angeht, beruht die Konzeption auf der Node Turtlesim~\footnote{\url{http://wiki.ros.org/turtlesim}} von ROS~(siehe~\autoref{subsec:Grundlagen_Turtlesim}).

\section{Der Schwarm}


\subsection*{Grundstruktur}

\subsection*{Einfluss verschiedener Parameter}

\section{Anführer}

\subsection*{Generelles Verhalten}

\subsection*{Verlieren des Schwarms}