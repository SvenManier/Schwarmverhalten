\chapter{Evaluation}\label{ch:Evaluation}

In diesem Kapitel werde ich die Ergebnisse der einzelnen Phasen der Thesis beschreiben. Nach jeder Phase wurden Simulationen mit der Implementierung gestartet und verschiedene Informationen aufgezeichnet. Der Hauptgrund hierfür ist, dass durch die Ergebnisse zum einen ein allgemeines Gefühl für den Schwarm gegeben werden kann und wie er einzustellen ist. Zum anderen liefern die Beobachtungen wichtige Informationen für die späteren Phasen der Konzeption.
In diesem Kaputel werden nicht alle Statistiken vorgestellt, sondern nur die, die als interessant empfunden wurden.

\section{Generelles zur Evaluation}

Da die Roboter des Schwarms einen gewissen freien Willen haben, der entscheidet wo sie hinfahren, und dieser schlicht Zufall ist, werden die Simulationen mit ihren Parametern immer mehrmals durchgeführt. Um Ausreißer in der Auswertung auszumerzen, wurde ebenfalls nicht mit einem normalen Durchschnitt gerechnet, sondern mit einem Durchschnitt über die mittleren Ergebnisse.

Da die Simulationen mit über 5 Parameter weit mehr als 100 verschiedene Einstellungen zulassen, konnte nicht jede Konfiguration getestet werden. Stattdessen wurden die Einstellungen für die Simulationen durch Gefühl und den Ergebnissen der vorangegangenen Simulationen ausgewählt, um die Anzahl im Bereich des zeitlich machbaren zu halten.

\section{Nachbau des Schwarms nach Craig Reynolds}

\begin{wrapfigure}{r}{\pictureWidth}
	\includegraphics[width=\pictureWidth,keepaspectratio]{graphics/SchwarmEinordnung.png}
	\caption{Einordnung der Roboter in Schwärme}
	\label{pic:SchwarmEinordnung}
\end{wrapfigure}

Nachdem die Implementierung des Schwarms abgeschlossen ist, galt es diese zu testen und Statistiken zu erheben. Dazu wurde die Simulation mit verschiedenen Parametern gestartet und 25 Roboter zufällig in der Simulation verteilt. Die Simulationen wurden für 1 Stunde laufen gelassen und in 10s-Abständen gemessen, wie viele Schwärme sich gebildet haben.
Ein Schwarm war hierbei eine zusammenhängende Kette von Robotern, die nicht weiter als ihre lokale Reichweite voneinander entfernt sind. Das bedeutet, dass alle Roboter innerhalb eines Schwarm die anderen beeinflussen. Teilweise mag dies direkt geschehen sein, wenn sie innerhalb der lokalen Reichweite waren. Aber auch indirekte Beeinflussung ist es möglich, indem Roboter beeinflusst wurden, durch Roboter die von anderen beeinflusst wurden.

\autoref{pic:SchwarmEinordnung} zeigt ein Beispiel mit 2 Schwärmen. Die Roboter sind als blaue Kreise eingezeichnet, ihre jeweilige lokale Reichweite als schwarzer Kreis drum herum. Eine Zahl im inneren zeigt die Zugehörigkeit zum jeweiligen Schwarm.

Da das Verhalten der Roboter, aufgrund ihres freien Willens und der zufälligen Platzierung in der Simulation, starken Schwankungen ausgesetzt ist, wurden immer 10 Simulationen gestartet und aus den mittleren 6 Werten ein Durchschnitt gebildet. Dadurch wurden Ausreißer eliminiert und die Statistiken können sinnvolle Mittelwerte zeigen. Der Rand der Simulation wurde als Hinderniss gewertet, was bedeutet, dass die Roboter nicht stumpf weiter gefahren, sind nachdem dieser erreicht wurde, sondern sie versucht haben diesem auszuweichen.

\subsection*{Statistik: Freier Wille}\label{subsubsec:StatistikFreierWille}

In~\autoref{pic:GeneralFlockStatistic1} zu sehen, ist eine Statistik die die Entwicklung von Schwärmen mit verändertem freien Willen zeigt. Über die X-Achse hinweg ist die Zeit der Simulation von 0-60 Minuten, die Y-Achse zeigt die Anzahl der vorhandenen Schwärme nach obiger Definition. Grundsätzlich wäre es innerhalb der Simulation aufgrund des verfügbaren Platzes möglich gewesen, dass jeder Roboter nur sich selbst als Schwarm hat.
Neben den Graphen selbst, die die Entwicklung der Schwärme zeigen, ist ebenfalls ein linearer Trend in gleicher Farbe und gestrichelter Linie eingezeichnet worden.

\paragraph*{Eigenschaften des Schwarms:}
\begin{itemize}
	\item Größe des Schwarms: 25 Roboter
	\item Lokale Reichweite: 5\per der Größe der Simulation
	\item Geschwindigkeit: 10\per der lokalen Reichweite
	\item Drang zur Gruppierung: 0\per
	\item Freier Wille: Variabel
\end{itemize}

Zu sehen ist, dass ein Roboter mit einem freien Willen von 225° (112.5° in beide Richtungen) eher dazu neigt sich zu verteilen statt sich zu sammeln. Der Einfluss der Regel sich nach seinen Nachbarn auszurichten liegt bei unter 50\per und unterliegt somit dem freien Willen. Dass es dennoch dazu kam dass sich Schwärme gebildet haben, resultiert einerseits daraus, dass aus der Menge [-112.5°, 112.5°] der letztliche Wert zufällig ermittelt wurde, der Durchschnitt also um 0° herum liegt, mit einer durchschnittlichen Abweichung von 56.25° zu beiden Seiten. Anderseits spielt aber auch der mangelde Platz zum verteilen eine Rolle und Roboter die zufällig aufeinander zugefahren sind im nächsten Tick als Schwarm erkannt wurden, unabhängig vom Grund warum sie zusammen waren.

\begin{figure}
	\includegraphics[width=\textwidth,height=\statisticHeight]{graphics/Statistics/FlockGeneral/LocalRange1Speed01ToFlock0.png}
	\caption{Entwicklung der Schwärme in Abhängigkeit ihres freien Willens}
	\label{pic:GeneralFlockStatistic1}
\end{figure}

Bei einem freien Willen von 180° und 135° zeigt sich der Trend nahezu konstant. Schwärme wurden gebildet und im gleichen Maße wieder aufgelöst.

Ab einem freien Willen von 90° und darunter zeigt sich dagegen ein Trend zur Gruppierung. Der Drang zur Gruppierung nimmt einen Großteil des Verhaltens ein und auf lange Sicht wäre die Zahl der Schwärme gegen 1 gesunken. Dies zeigt sich vor allem bei einem Wert von 45°, der bereits während einiger der einstündigen Simulationen dazu geführt hat, dass sich alle Roboter zu einem einzigen Schwarm versammelt haben.

\subsection*{Statistik: Gruppierungsdrang}

\begin{figure}
	\includegraphics[width=\textwidth, height=\statisticHeight]{graphics/Statistics/FlockGeneral/LocalRange1Speed01FreeWill90.png}
	\caption{Entwicklung der Schwärme in Abhängigkeit ihres Dranges zur Gruppierung mit 90° eigenem Willen}
	\label{pic:GeneralFlockStatistic2}
\end{figure}

In~\autoref{pic:GeneralFlockStatistic2} und~\autoref{pic:GeneralFlockStatistic3} zu sehen, sind Statistiken die die Entwicklung von Schwärmen mit verändertem Drang zur Gruppierung zeigen. Über die X-Achse hinweg ist die Zeit der Simulation von 0-60 Minuten, die Y-Achse zeigt die Anzahl der vorhandenen Schwärme. In der ersten Statistik behielten sie dabei einen freien Willen von 90°, bei der zweiten wurde ein freier Wille von 225° eingestellt.

\paragraph*{Eigenschaften des Schwarms:}
\begin{itemize}
	\item Größe des Schwarms: 25 Roboter
	\item Lokale Reichweite: 5\per der Größe der Simulation
	\item Geschwindigkeit: 10\per der lokalen Reichweite
	\item Freier Wille: 90° bzw. 225°
	\item Drang zur Gruppierung: Variabel
\end{itemize}

Bei einem freien Willen von 90° ist zu sehen, dass die Anzahl der Schwärme bei einem Drang zur Gruppierung von 0\per konstant bleibt. Diese Entwicklung war bereits in \autoref{subsubsec:StatistikFreierWille} zu sehen. Die Start-Anzahl war dabei in allen Simulationen annähern gleich.
Sobald der Drang zur Gruppierung über 0\per steigt, zeigt sich, dass die Roboter sich abhängig vom eingestellten Wert unterschiedlich schnell zu Schwärmen zusammenziehen und diese auch nicht wieder verlassen. Ein höherer Wert sorgte dabei analog dafür, dass sich die Schwärme schneller sammelten.

\begin{figure}
	\includegraphics[width=\textwidth, height=\statisticHeight]{graphics/Statistics/FlockGeneral/LocalRange1Speed01FreeWill225.png}
	\caption{Entwicklung der Schwärme in Abhängigkeit ihres Dranges zur Gruppierung mit 225° eigenem Willen}
	\label{pic:GeneralFlockStatistic3}
\end{figure}

Bei einem freien Willen von 225° zeigt sich dagegen deutlich mehr Varianz. Die Entwicklung um 0\per herum ist dabei wieder die gleiche wie in \autoref{subsubsec:StatistikFreierWille} bereits zu sehen war. Ein Wert von 5\per neigt die Trendlinie zwar leicht nach unten, die Daten zeigen aber nach wie vor hohe Fluktuationen und Schwärme die sich zunächst gebildet haben, brechen aufgrund des großen freien Willens wieder auseinander und die Roboter trennen sich. Bei 10\per wird der Trend zwar umso deutlicher, die Fluktuationen sind aber nach wie vor zu sehen und Schwärme neigen noch immer dazu sich gelegentlich zu trennen. Erst ab einem Wert von 50\per bleiben Schwärme stabil und langsam aber sicher fügen sich alle Roboter zu einem einzelnen Schwarm zusammen.

\subsection*{Statistik: Negativer Gruppierungsdrang}

Zum Schluss wurde noch eine Statistik mit negativem Gruppierungsdrang erstellt. Hintergrund dieses Versuchs war es, Roboter verteilen zu lassen. Damit sie sich während des Leerlaufs nicht versammeln und so in ihrer Masse zu einem größeren Hinderniss werden, wurde versucht die Roboter dazu zu bringen sich zu meiden und damit auszulösen, dass sie sich verteilen.

\paragraph*{Eigenschaften des Schwarms:}
\begin{itemize}
	\item Größe des Schwarms: 25 Roboter
	\item Lokale Reichweite: 5\per der Größe der Simulation
	\item Geschwindigkeit: 10\per der lokalen Reichweite
	\item Freier Wille: 45°
	\item Drang zur Gruppierung: Variabel
\end{itemize}

Wie in \autoref{pic:NegativerGruppierungsdrang} zu sehen ist, versuchen sich die Roboter bei einem  freien Willen von 45° und einem Gruppierungsdrang von 0\per zu Schwärmen zusammenzuschließen. Über die X-Achse hinweg ist die Zeit der Simulation von 0-60 Minuten, die Y-Achse zeigt die Anzahl der vorhandenen Schwärme. Dieses Verhalten ist nachvollziehbar, da ein derart geringer freier Wille dazu führt, dass die Bewegung größenteils davon abhängig ist die anderen zu imitieren. Roboter im Radius ihrer lokalen Reichweite fahren also meist in die selbe Richtung und gruppieren sich so auf natürliche Weise. Wenn sie gegen die Grenzen der Simulation stoßen und versuchen dieser auszuweichen, kommen sie dabei noch näher zusammen.

Ein negativer Gruppierungsdrang von bereits 0.05\per reicht jedoch aus um die Trendlinie merklich nach oben zu verlagern und die Gruppierung in Schwärme zumindest langsamer vollziehen zu lassen. Die Fluktuationen nehmen deutlich zu und die Schwärme scheinen nun deutlich instabiler zu sein. Zwar finden sich immer wieder Roboter in Schwärmen ein, sie brechen aber auch oft auseinander und die Einheiten gehen wieder getrennte Wege.

\begin{figure}
	\includegraphics[width=\textwidth, height=\statisticHeight]{graphics/Statistics/FlockGeneral/LocalRange1Speed01FreeWill45NegativeToFlock.png}
	\caption{Entwicklung der Schwärme mit negativem Gruppierungsdrang}
	\label{pic:NegativerGruppierungsdrang}
\end{figure}

Schon ab 1\per negativen Gruppendrangs zeigt die Trendlinie, dass die Gruppen eher dazu neigen sich aufzulösen statt sich neu zu bilden. Der Winkel zur Mitte des eigenen Schwarms kann maximal 180° (in beide Richtungen) groß sein, das heißt der negative freie Wille kann einen Einfluss von maximal 1.8° einnehmen. Trotzdem reicht bereits dieser Einfluss aus um die Roboter zu zu bewegen sich voneinander weg zu bewegen.

Ab einem Wert von 10\per zeigt sich letztlich eine totale Streuung und die Trendlinie bleibt eine Konstante. Die Roboter schaffen es teilweise keinen einzigen Schwarm zu bilden (bzw. nur Schwärme in denen sie selbst als einziger Roboter vertreten sind).

Überraschenderweise zeigt ab einem Wert von 50\per die Trendlinie wieder nach unten und die Roboter scheinen sich erneut zu gruppieren. Der Grund hierführ ist, dass die Roboter beim Ausweichen ihrer Kollegen überschießen und sich so weit in die andere Richtung drehen, dass sie letztlich vom Winkel wieder näher dran sind als vorher. Legt man Wert darauf, dass sich die Roboter verteilen, gilt es also einen passenden Prozentwert einzustellen und diesen nicht zu übertreiben, da man sonst das Gegenteil dessen erreicht was man als Ziel hatte.


\section{Anführer}

Nachdem der Anführer implementiert wurde, wurde das konzipierte Verhalten getestet und verschiedene Statistiken erhoben, inwiefern ein Anführer in der Lage ist einen Schwarm an sein Ziel zu führen, ohne dass diese aktiv gesteuert werden müssen.
Dafür wurde der gesamte Schwarm an einer Ecke der Simulation platziert. Die Einheiten waren dabei immer nahe genug beieinander, um als einzelner Schwarm angesehen zu werden, aber ansonsten zufällig verteilt. Einem der Roboter wurde daraufhin zufällig eine \highlight{New\_Mission} zugeteilt und damit zum Anführer gewählt. Dieser hat daraufhin versucht seinen Schwarm zum Ziel zu führen.

Auch in dieser Analyse spielt der Zufall eine beachtliche Rolle, da der freie Wille der Einheiten darauf ausgelegt ist. Es wurden daher immer 10 Simulationen durchgeführt und aus den mittleren 6 Ergebnissen ein Mittelwert berechnet, um Aureißer möglichst zu ignorieren.

\subsection*{Nachbesserung der Implementierung}\label{subsec:AnalyseNachbesserung}

Recht schnell hat sich gezeigt, dass das Einfangen des Schwarms, wie es in \autoref{subsubsec:AnführerEinfangen} vorgestellt wurde, mehr negative als positive Auswirkungen hat. Versucht der Roboter seinen Schwarm einzufangen, ändert er, dadurch dass er sich um 180° umdrehen muss, die gesamte Ausrichtung des Schwarms. Dies führte in den Simulationen meist dazu, dass der Schwarm vollkommen abgedriftet ist und der Anführer Mühe und Not hatte, ihn wieder auf das Ziel auszurichten. Meist verlor er dabei den Schwarm wieder, bevor dieser ausgerichtet war, womit das ganze wieder von vorne los ging. Durch dieses Fehlverhalten mussten die meisten Simulationen abgebrochen werden und die Ergebnisse waren letztlich unbrauchbar, da zu viel von Hand gefiltert werden musste.

Aus diesem Grund wurde der Teil des Algorithmus' wieder entfernt. Stattdessen wartet der Anführer nun, wenn der Schwarm 50\per der lokalen Reichweite entfernt ist und gibt auf, wenn der Mittelpunkt die lokale Reichweite ganz verlassen hat. Aufgeben bedeutet in diesem Kontext, dass der Anführer ohne seinen Schwarm zum Ziel gefahren ist. Dies führte allgemein zu besseren Ergebnissen und der Schwarm konnte zumindest in Teilen zum Ziel gebracht werden.

\subsection*{Abhängigkeit: Freier Wille}\label{subsubsec:AbhängigkeitFreierWille}

In dieser Statistik wurde geprüft wie sich verschiedene Werte für den freien Willen darauf auswirken, dass der Schwarm zum Ziel geführt werden kann. Über die X-Achse von \autoref{pic:LeaderDependencyFreeWill} hinweg ist die vergangene Wegstrecke in Prozent angegeben, die Y-Achse zeigt die Anzahl der Roboter die im Schwarm des Anführers vorhanden sind.

\paragraph*{Eigenschaften des Schwarms:}
\begin{itemize}
	\item Größe des Schwarms: 25 Roboter
	\item Anzahl der Anführer: 1 Anführer
	\item Lokale Reichweite: 5\per der Größe der Simulation
	\item Geschwindigkeit: 10\per der lokalen Reichweite
	\item Drang zur Gruppierung: 5\per
	\item Freier Wille: Variabel
\end{itemize}

In der \autoref{pic:LeaderDependencyFreeWill} zu sehen, ist dass der Schwarm anfangs immer zusammen blieb. Da die Roboter als gemeinsamer Schwarm zusammen gestartet sind, mussten sie sich nicht erst finden. Wenn der Schwarm in eine andere Richtung lenkte als der Anführer, blieb der Anführer ab einer bestimmten Entfernung stehen und wartete. Zog der Schwarm weiterhin in die Richtung löste er sich irgendwann vom Anführer und dieser zog alleine weiter Richtung Ziel. Durch den relativ hohen Wert von 5\per im Gruppendrang blieb der Schwarm jederzeit zusammen. Wenn sich der Schwarm vom Roboter löste, tat er dies immer im gesamten.

\begin{figure}
	\includegraphics[width=\textwidth, height=\statisticHeight]{graphics/Statistics/Leader/DependencyFreeWill.png}
	\caption{Erfolg eines Anführer ins Abhängigkeit des freien Willens}
	\label{pic:LeaderDependencyFreeWill}
\end{figure}

Je niedriger der Wert im freien Willen war, desto eher löste sich der Schwarm scheinbar von seinem Anführer. Dies ist darauf zurückzufiühren, dass bei einer Gruppengröße von 25 Robotern der Einfluss eines Anführers nur 4.2\per ausmacht und der Schwarm mehr vom Zufall als vom Anführer gelenkt wird. Dass die Roboter mit dem größeren freien Willen letztlich scheinbar länger beim Anführer blieben, ist dem geschuldet, dass die Roboter weniger eng als Schwarm zusammen bleiben und durch die Verteilung der Roboter mehr in der Nähe des Anführers blieben. Die Roboter mit dem geringen freien Willen hingegen blieben mehr zusammen und sind somit 
%
%\subsubsection*{Abhängigkeit: Gruppenzugehörigkeit}
%
%In den folgenden Versuchen wurde geprüft, wie sich die Erfolgsquote des Anführers verändert, wenn statt des freien Willens der Gruppendrang variabel gehalten wird. Über die X-Achse von \autoref{pic:LeaderDependencyToFlockFreeWill90} und \autoref{pic:LeaderDependencyToFlockFreeWill180} hinweg ist die vergangene Wegstrecke in Prozent angegeben, die Y-Achse zeigt die Anzahl der Roboter die im Schwarm des Anführers vorhanden sind.
%
%\paragraph*{Eigenschaften des Schwarms:}
%\begin{itemize}
%	\item Größe des Schwarms: 25 Roboter
%	\item Anzahl der Anführer: 1 Anführer
%	\item Lokale Reichweite: 5\per der Größe der Simulation
%	\item Geschwindigkeit: 10\per der lokalen Reichweite
%	\item Drang zur Gruppierung: Variabel
%	\item Freier Wille: 90° bzw. 180°
%\end{itemize}
%
%Bei einem Gruppendrang von 1\per zeigt sich, ähnlich zu \autoref{subsubsec:AbhängigkeitFreierWille}, dass ein größerer Wert dazu führt, dass die Roboter eher unter sich bleiben und sich weniger streuen. Der Anführer hat mit seinen 4.2\per nicht genug Einfluss um den Schwarm bei sich zu halten und ab einem bestimmten Punkt entschließt sich der Schwarm dazu einen anderen Weg einzunehmen.
%
%\begin{figure}
%	\includegraphics[width=\textwidth, height=\statisticHeight]{graphics/Statistics/Leader/DependencyToFlockFreeWill90_new.png}
%	\caption{Erfolg eines Anführer ins Abhängigkeit der Gruppenzugehörigkeit mit einem freien Willen von 90°}
%	\label{pic:LeaderDependencyToFlockFreeWill90}
%\end{figure}
%
%Auch ähnlich sind die Ergebnisse bei einem freien Willen von 180°. Die Roboter streuen sich mehr und bilden generell mehr Schwärme. Dadurch bleiben automatisch auch mehr Roboter in der Nähe des Anführers, weswegen er es letztlich scheinbar schafft mehr Roboter mit ins Ziel zu führen.
%
%\begin{figure}
%	\includegraphics[width=\textwidth, height=\statisticHeight]{graphics/Statistics/Leader/DependencyToFlockFreeWill180_new.png}
%	\caption{Erfolg eines Anführer ins Abhängigkeit der Gruppenzugehörigkeit mit einem freien Willen von 180°}
%	\label{pic:LeaderDependencyToFlockFreeWill180}
%\end{figure}

\subsection*{Abhängigkeit: Gruppengröße und -drang}

In den folgenden Statistiken wurde ein Anführer mit steigender Gruppenzahl und unterschiedlichem Gruppendrang gemessen. Über die X-Achse von \autoref{pic:LeaderDependencyFreeWill} hinweg ist der genutzte Gruppendrang angegeben.
Das erste Diagramm zeigt dabei, für wie viel Prozent der Wegstrecke der gesamte Schwarm beisammen gehalten werden konnte.
Das zweite Diagramm zeigt die Größe des Schwarms um den Anführer herum beim erreichen des Ziels.
Das dritte Diagramm zeigt die Anzahl der Wartezeiten die der Anführer hatte. Eine höhere Anzahl an Wartezeiten ist direkt equivalent zu einer längeren Zeit die die Simulation gebraucht hat.

Dabei werden immer 2 Werte gezeigt. 'Mittelwert', der einen einfachen Mittelwert über alle 15 Messwerte anzeigt und 'Median', bei dem der Mittelwert auf die mittleren 10 Messwerte begrenzt wurde um Ausreißer auszusortieren. Da der Anführer selbst immer am Ziel ankommt, wurde er bei den Messwerten abgezogen. Die Anzahl der dargestellten Roboter entspricht also immer denjenigen, die passiv zum Ziel geführt wurden.

\paragraph*{Eigenschaften des Schwarms:}
\begin{itemize}
	\item Größe des Schwarms: Variabel Roboter
	\item Anzahl der Anführer: 1 Anführer
	\item Lokale Reichweite: 5\per der Größe der Simulation
	\item Geschwindigkeit: 10\per der lokalen Reichweite
	\item Drang zur Gruppierung: Variabel
	\item Freier Wille: 90°
\end{itemize}

\begin{figure}
	\includegraphics[width=4.9cm, height=4cm]{graphics/Statistics/Leader/FlockSize/5_1.png}
	\includegraphics[width=4.9cm, height=4cm]{graphics/Statistics/Leader/FlockSize/5_2.png}
	\includegraphics[width=4.9cm, height=4cm]{graphics/Statistics/Leader/FlockSize/5_3.png}
	\caption{Erfolg eines Anführer ins Abhängigkeit der Gruppengröße und des Gruppendrangs mit 5 passiven Robotern}
	\label{pic:LeaderSize5}
\end{figure}

In \autoref{pic:LeaderSize5} zu sehen ist zunächst einmal, dass der Anführer in einer Gruppe mit nur 5 Robotern einen sehr starken Einfluss auf die Gruppe hat. Bereits ohne Gruppendrang gibt es eine Erfolgsquote von über 20\%. Ab 5\per Gruppendrang gibt es schon annähernd 80\per Erfolgsquote und ab 10\per bekommt er seinen Schwarm immer ans Ziel gebracht.

Die Größe des Schwarms zeigt ein sehr ähnliches Bild. Kommen bei 0\per Gruppendrang im Schnitt nur sehr wenige Roboter an, steigert es sich sprunghaft auf 4 Einheiten und ab einem Gruppendrang von 10\per kommen immer alle Roboter an.

Die Wartezeiten sind nicht ganz so konstant, zeigen aber, dass es noch deutliche Unterschiede zwischen den Prozentwerten im Bereich von 10-100\per gibt, auch wenn die beiden vorigen Diagramme dort keine mehr gezeigt haben. Mit einem höheren Gruppendrang sinkt die Anzahl der Wartezeiten immer mehr und bei 100\per gibt es sogar annähernd gar keine mehr.

\begin{figure}
	\includegraphics[width=4.9cm, height=4cm]{graphics/Statistics/Leader/FlockSize/10_1.png}
	\includegraphics[width=4.9cm, height=4cm]{graphics/Statistics/Leader/FlockSize/10_2.png}
	\includegraphics[width=4.9cm, height=4cm]{graphics/Statistics/Leader/FlockSize/10_3.png}
	\caption{Erfolg eines Anführer ins Abhängigkeit der Gruppengröße und des Gruppendrangs mit 10 passiven Robotern}
	\label{pic:LeaderSize10}
\end{figure}

Die Diagramme in \autoref{pic:LeaderSize10} zeigen, dass der Einfluss des Anführers mit 5 Robotern mehr schon deutlich nachlässt. Konnten vorher scon ab 10\per Gruppendrang beste Ergebnisse erziehlt werden, zeigt der Durchschnitt nun deutlich niedrigere Werte. Dies bedeutet vor allem dass die Fehlerquote zunahm und die 10 mittleren Ergebnisse nun ebenfalls höhere Schwankungen aufweisen. Nur bei 40\per sieht man noch einen vollen Erfolg über alle Messwerte hinweg.

Die Größe des Schwarms ist von ähnlichen Rückschlägen betroffen. Bei 0\per Gruppendrang schafft es nur jeder zehnte Roboter ins Ziel, fast der gleiche Wert wie bei 5 passiven Robotern. Bei 10\per Gruppendrang schafft es der Median auf einen vollen Erfolg, der Mittelwert hingegen erst ab 20\%.

Die Wartezeiten zeigen hingegen nicht unbedingt den gleichen Trend. Sie sind zwar allgemein stiegen, der Trend ist aber nicht mehr so steil wie in der Statistik mit 5 passiven Robotern. Steigen beide Diagramme bei ca. 30 Wartezeiten ein, ist nun auch bei 40\per Gruppendrang eine deutlich höhere Wartezeit zu sehen.

\begin{figure}
	\includegraphics[width=4.9cm, height=4cm]{graphics/Statistics/Leader/FlockSize/15_1.png}
	\includegraphics[width=4.9cm, height=4cm]{graphics/Statistics/Leader/FlockSize/15_2.png}
	\includegraphics[width=4.9cm, height=4cm]{graphics/Statistics/Leader/FlockSize/15_3.png}
	\caption{Erfolg eines Anführer ins Abhängigkeit der Gruppengröße und des Gruppendrangs mit 15 passiven Robotern}
	\label{pic:LeaderSize15}
\end{figure}

In \autoref{pic:LeaderSize15} nimmt der allgemeine Trend weiter seinen Lauf. Die Erfolgsquote für die Werte unter 40\per nehmen immer weiter ab, wenn auch die 40\per selbst noch einen vollen Erfolg vorweisen kann.

Es kommen auch weniger Roboter insgesamt im Ziel an, wenn die Gesamtzahl der ankommenden Roboter auch bei 20\per Gruppendrang noch recht hoch ist. Bei 40\per Gruppendrang ist nach wie vor ein voller Ausschlag zu sehen.

Auch die Wartezeiten zeigen den selben Trend wie zuvor. Sie starten recht hoch, der Trend bleibt aber flacher als bei den Messungen mit 5 passiven Robotern weniger.

\begin{figure}
	\includegraphics[width=4.9cm, height=4cm]{graphics/Statistics/Leader/FlockSize/20_1.png}
	\includegraphics[width=4.9cm, height=4cm]{graphics/Statistics/Leader/FlockSize/20_2.png}
	\includegraphics[width=4.9cm, height=4cm]{graphics/Statistics/Leader/FlockSize/20_3.png}
	\caption{Erfolg eines Anführer ins Abhängigkeit der Gruppengröße und des Gruppendrangs mit 20 passiven Robotern}
	\label{pic:LeaderSize20}
\end{figure}

Ab einer passiven Anzahl von 20 Robotern zeigt sich in \autoref{pic:LeaderSize20} erstmals der Trend, dass die Wartezeiten zunehmen, wenn der Gruppendrang größer wird. Ein genauerer Blick auf das Diagram zeigt jedoch auch, dass die Wartezeiten allgemein sehr viel niedriger sind als in den Messungen mit weniger Robotern. Dies ist vor allem darauf zurückzuführen, dass der Anfüher ab einer Zahl von 20 passiven Robotern einen Schwarm viel schneller vollständig verliert und es keine Wartezeiten mehr gibt, weil er schneller dazu übergeht das Ziel alleine aufzusuchen. Auch die anderen beiden Diagramme zeigen nun bei 40\per Gruppendrang keinen vollen Erfolg mehr.

\begin{figure}
	\includegraphics[width=4.9cm, height=4cm]{graphics/Statistics/Leader/FlockSize/25_1.png}
	\includegraphics[width=4.9cm, height=4cm]{graphics/Statistics/Leader/FlockSize/25_2.png}
	\includegraphics[width=4.9cm, height=4cm]{graphics/Statistics/Leader/FlockSize/25_3.png}
	\caption{Erfolg eines Anführer ins Abhängigkeit der Gruppengröße und des Gruppendrangs mit 25 passiven Robotern}
	\label{pic:LeaderSize25}
\end{figure}

In den letzten Diagrammen (\autoref{pic:LeaderSize25}) ist letztlich zu sehen dass sich der allgemeine Trend weiter fort führt. Die Wartezeiten sinken noch deutlicher, weil der Anführer seinen Schwarm immer früher verliert. Die erfolgreiche Strecke nimmt genau wie die Anzahl der Roboter die im Ziel ankommen immer mehr ab. Nur bei 40\per Gruppendrang bleiben die Werte noch immer recht hoch.











\section{Transport von Waren mit Hilfe eines Schwarms}

Das letztliche Ziel dieser Arbeit ist es zu prüfen, ob es möglich, und sinnvoll, ist Gegenstände mit Hilfe eines autonomen Schwarms zu transportieren. Dazu musste der Schwarm nun dazu gebracht werden Waren zu bewegen ohne die einzelnen Roboter zu sehr zu beeinflussen. Da generell die Roboter nicht zentral gesteuert werden sollen und auch die Logik möglichst simple bleiben soll, sollte das Programm der einzelnen Einheiten möglichst wenig verändert werden und der 'natürliche' Trieb der Einheiten genutzt werden.

\subsection*{Erteilen von Aufträgen}

Eine der ersten Dinge im Ablauf eines Transports ist die Erteilung des Auftrags. Dazu wurde ein neuer Nachrichten-Typ Namens 'MissionNew' im ROS-Netzwerk eingeführt der die folgenden Felder hat:

\begin{lstlisting}[frame=L]
uint8 robot_index_from
uint8 robot_index_to

uint8 leader_number
uint8 mission_id

float32 object_position_x
float32 object_position_y

float32 object_size_x
float32 object_size_y

float32 target_x
float32 target_y
\end{lstlisting}

Der Nachrichten-Typ definiert eine Spanne von Robotern die den Auftrag ausführen sollen. Diese werden mit ihren IDs angesprochen. Das Intervall der IDs ist, gemäß Programmierstandards, als halboffenes Intervall definiert: [robot\_index\_from, robot\_index\_to[.
leader\_number gibt die Anzahl der Leader an die verwendet werden sollen und mission\_id gibt dem derzeitigen Auftrag eine fixe ID um diesen und zugehörige Dinge genau identifizieren und verbinden zu können.
Mit object\_position\_x/-\_y ist die Startposition des zu transportierenden Objektes angegeben. Zusammen mit object\_size\_x/-\_y, welche die Ausmaße des Objektes angeben. Die Position des Objekts ist als Mitte des Objekts definiert.
target\_x/-\_y gibt die letztliche Position an die das Objekt einnehmen soll.

Der Auftrag wird von außen an das Topic 'flock/mission/new' gesendet. Der Ersteller des Auftrags muss keine ROS-Node sein, auch wenn dies verschiedene Vorteile hätte. Dadurch das ROS-Topics auch über das Terminal engesprochen werden können, ist das Senden der Nachrichten auch über jedes andere Programm möglich. Das Topic für die neuen Missionen wird von jedem aktiven Roboter abonniert. Entsprechend nimmt jeder Roboter Notiz von diesem Auftrag, auch wenn er nicht direkt mit seiner ID angesprochen wird.

\subsection*{Von Gefahrenzonen zu Sicherheitszonen}

Um die betreffenden Roboter in die Zone zu senden, auf der ihnen später das Transportobjekt aufgesetzt wird, wird ein Mechanismus verwendet der bereits vorher Einzug in das ROS-System hielt: Gefahrenzonen. Diese wurden leicht angepasst um sie invertieren zu können und somit aus einer Gefahrenzone eine Sicherheitszone zu machen. Ist ein Bereich als Sicherheitszone definiert, ist jeder andere Bereich automatisch eine Gefahrenzone.
Dieser Mechanismus birgt nur eine kleine Änderung im System, schafft es aber Roboter in einen bestimmten Bereich zu locken ohne sie aktiv steuern zu mpsse, indem einfach ihr 'natürlicher' Trieb verwendet wird, von Gefahren zu flüchten.

Wird ein Auftrag erteilt, so nehmen die Roboter die dem Auftrag zugeteilt sind, eine neue Sicherheitszone auf, die den Ausmaßen und der Position des Transportobjekts am Aufnahmeort entsprechen. Roboter die nicht dem Transport zugeteilt wurden, werden diese neue Zone als Gefahrenzone auffassen und versuchen sich diesem Gebiet fernzuhalten.

Ein Objekt kann grundsätzlich aus verschiedenen Geometrien zusammengesetzt werden. Dadurch ist es möglich nicht nur Objekte zu transportieren die eine einfache Form wie Rechtecke oder Kreise haben, sondern verschiedene Rechtecke können dann zu einem 'L' oder 'U' zusammengesetzt werden.

\begin{wrapfigure}{r}{\pictureWidth}
	\includegraphics[width=\pictureWidth,keepaspectratio]{graphics/Klassendiagramme/KlassendiagrammObject.png}
	\caption{Die Klassenhierarchie des Objekt-Systems}
	\label{pic:KlassendiagrammObject}
\end{wrapfigure}

\paragraph*{Implementierung}
Die Klassenhierarchie ist wie in \autoref{pic:KlassendiagrammObject} dargestellt. Die Implementierung eines Objekts ist als Sammelklasse definiert, die verschiedene Geometrien unter sich vereint. Die Geometrien haben eine Hauptklasse von der sie sich ableiten. Die abgeleiteten Klassen müssen letztlich nur noch definieren, wann etwas innerhalb ihrer Fläche ist und was ihre Ausmaße im zweidimensionalen Raum sind. Wird ein Objekt auf eventuelle Kollisionen abgefragt, muss dieses letztlich nur noch über die inneliegenden Geometrien iterieren und sobald eine Kollision statt fand ist das Gesamtergebnis ebenfalls eine Kollision.

\subsection*{Füllen eines Raums}

Wird eine Sicherheitszone definiert, versuchen die Roboter, die sich in der Gefahrenzone befinden, auf möglichst schnellstem Wege die Sicherheitszone zu betreten. Dazu wird das Zentrum der Sicherheitszone ins Ziel genommen und (ohne mit anderen Robotern zu kollidieren) der direkte Weg darauf zu genommen. Dabei kann es dann allerdings vorkommen, dass eine komplexere Form nicht vernünftig gefüllt werden kann, oder dies sehr lange dauert. Da ein eintreffender Roboter so lange auf den Mittelpunkt zufahren würde, bis die Roboter darin sich genug verteilt haben um genug Platz für den neuen Roboter zu schaffen.

Ein Algorithmus wie man es schafft mit Roboter einen definierten Raum zu füllen lässt sich in (\note{HIER; QUELLE}) finden. Dieser Algorithmus lässt sich aufgrund anderer Fähigkeiten bei den Robotern nicht und dem gewünschten Schwarmverhalten nicht vollständig nachbilden. Eine andere, bereits implementierte Fähigkeit, lässt sich dagegen gut nutzen um den Algorithmus annähernd nachzubauen: Ausweichen.

\begin{wrapfigure}{r}{\pictureWidth}
	\includegraphics[width=\pictureWidth,keepaspectratio]{graphics/AlgorithmusAusweichenTransport.png}
	\caption{Roboter füllen das Transportobjekt}
	\label{pic:AlgorithmusAusweichenObjekt}
\end{wrapfigure}

Betritt ein Roboter die Sicherheitszone des Transport-Objekts, reduziert dieser seine Geschwindigkeit und fährt langsam weiter auf den Mittelpunkt zu. Neu ankommende Roboter werden durch die vorderen, viel langsameren Roboter zum Ausweichen gezwungen, was letztlich dazu führt, dass die Roboter sich seitlich verteilen, aber langfristig den Nachfolgern Platz machen. Ist ein Roboter nahe genug am Mittelpunkt angekommen oder findet in einem Umkreis von [-90°, 90°] keinen Platz mehr der näher am Mittelpunkt ist als der aktuelle, bleibt er stehen. Roboter die stehen bleiben senden den anderen Robotern ihres Schwarms ein entsprechendes Signal dass sie bereit sind. Haben alle Roboter erkannt dass die anderen bereit sind, ist die Phase des Eintreffens am Abnahmepunkt abgeschlossen.

Von hier an braucht es ein Event dass den Robotern zeigt dass sie mit dem Transportobjekt Richtung Ziel losfahren können. Möglich wäre eine Zeitsteuerung für streng automatisierte Prozesse. Aber auch interne Signale über ROS sind möglich. Durch die ID die jeder Auftrag hat, wäre eine einfache Nachricht '[TransportID\#][Losfahren]' bereits zielführend.

\paragraph*{Der Algorithmus am Beispiel}
\autoref{pic:AlgorithmusAusweichenObjekt} zeigt den Algorithmus anhand eines Beispiels mit 4 Robotern.

Der erste Roboter versucht zur Mitte des Objekts zu gelangen und findet sofort Zugang zum Objekt. Er reduziert seine Geschwindkeit und platziert sich langsam im Mittelpunkt des Objektes. Da seine Nähe zum Mittelpunkt klein genug ist, bleibt er letztlich stehen.

Der nächste Roboter kommt, wird allerdings von seinem Vorgänger leicht blockiert. Das führt dazu dass dieser versucht nach links auszuweichen und damit Erfolg hat. Er betritt die Zone, findet aber keinen besseren Ort und bleibt daraufhin stehen.

Der dritte Roboter wird ebenfalls von seinen Vorgängern blockiert, findet aber rechts eine Stelle. Nachdem er diese eingenommen hat findet auch er keine bessere Stelle und bleibt ebenfalls stehen.

Der vierte Roboter findet zunächst links eine Stelle und betritt daraufhin die Zone. Er verlangsamt seine Bewegung und sucht nun nach Stellen die ihn ohne Kollisionen näher zu seinem Ziel führen, wobei er einmal Roboter\#2 umkreist. In der Lücke zwischen Roboter\#2 und Roboter\#1 findet er die beste Stelle und bleibt dort anschließend stehen. Der Algorithmus für diese 4 Roboter ist damit abgeschlossen. Sie stehen alle in der Fläche des Transportobjektes und stehen still, bereit die Lieferung entgegen zu nehmen.

\subsection*{Der Transport}

Um den Transport selbst zu realisieren bedient man sich der Hilfe der Anführer. Diese richten sich dynamisch nach dem Winkel aus, den das Transportobjekt zum Ziel hat, wie \autoref{pic:AlgorithmusTransport} zeigt. Danach fangen sie an sich langsam in die Richtung zu bewegen in die sie sich ausgerichtet haben. Die anderen Roboter werden sich aufgrund des Gruppenverhaltens ebenfalls mehr oder weniger nach ihren Anführern ausrichten und das Objekt so langsam Richtung Ziel bewegen.

\begin{wrapfigure}{r}{\pictureWidth}
	\includegraphics[width=\pictureWidth,keepaspectratio]{graphics/AlgorithmusTransport.png}
	\caption{Der Transport des Objekts}
	\label{pic:AlgorithmusTransport}
\end{wrapfigure}

\subsubsection*{Einhalten der Richtung hat Priorität}
Kann ein Anführer keine normale Bewegung ausführen, weil er sonst mit einem anderen Roboter kollidieren würde, darf er nicht versuchen auszuweichen, da dies sonst die Ausrichtung der passiven Roboter negativ beeinflussen könnte. Stattdessen drosselt er zunächst sein Bewegungstempo oder bleibt, falls notwendig, ganz stehen. Die richtige Richtung beizuhalten ist wichtig, da sich passive Roboter nicht unmittelbar nach ihren Anführern ausrichten. Gerade wenn es zahlenmäßig wenige Anführer im Vergleich zu passiven Robotern sind, kann es einige Zeit dauern, bis die Einheiten so weit beeinflusst wurden, dass sie in die gewünschte Richtung zeigen. Dreht sich ein Anführer in die verkehrte Richtung, vielleicht sogar in die entgegengesetzte, kann dies zu einer Kettenreaktion führen die alle Roboter betrifft und zusätzlich das Transportobjekt stark vom Weg abbringen.

\subsubsection*{Abschluss des Transports}






