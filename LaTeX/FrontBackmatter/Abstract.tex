%*******************************************************
% Abstract
%*******************************************************
\pdfbookmark[0]{Zusammenfassung}{Zusammenfassung}
\chapter*{Zusammenfassung}

In dieser Thesis wird geprüft ob, und wie, ein Schwarm aus autonomen Robotern in der Lage ist Transportaufträge zu erledigen. Dabei handelt es sich um Aufträge, die mehrere Robotern voraussetzen und nicht von einzelnen erledigt werden können. Außerdem unterliegen die Roboter dem Schwarmverhalten das von Craig Reynolds und beschrieben und skizziert wurde, das heißt sie orientieren sich am Gruppenverhalten von echten Tieren wie man es in der Natur bei großen Gruppen beobachtet hat. Die Kommunikation innerhalb des Schwarms beläuft sich demnach auf ein Minimum und soll eher fehlende Sensorik und Gestik/Mimik ausgleichen, als dass sie für Absprachen genutzt wird.

Um der Frage auf den Grund zu gehen, wurde in iterativen Schritten zunächst ein Verhalten erdacht, prototypisch umgesetzt und anschließend evaluiert, um die nächste Runde der Konzeption zu verbessern. Die Analyse hat letztlich gezeigt, dass Schwarmverhalten während eines Transports unerwünscht ist und Roboter die einer klaren Steuerung folgen deutlich besser agieren. Allerdings zeigte sich Schwarmverhalten außerhalb des Transports, also während es Standby-Betriebs oder auf dem Weg dorthin, als durchaus sinnvoll, da sich die Roboter ohne große Ressourcen dynamischer und natürlicher in ihre Umgebung einbinden.

Diese Thesis hat somit gezeigt, dass Schwarmverhalten nicht für Transportaufträge selbst geeignet ist, aber durchaus viel Potential zeigt, wenn es generell darum geht, dass sich Roboter einer Umgebung anpassen müssen, ohne zu Störobjekten zu werden.